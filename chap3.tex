\chapter{Aritmética Modular}
\label{chap:modular}


\section{A congruência módulo $m$}

Dado $m\in \NN$ fixado, dois inteiros $a$ e $b$ são
ditos \emph{congruentes} módulo $m$ se $m \mid b-a$. Denotamos
\index{Congruência módulo $m$}
essa relação por $a \equiv b \pmod m$.

\begin{example}\label{ex:modbas}
  $137 \equiv 61 \pmod{19}$ pois $19 \mid 137-61 = 76 = 4\times 19$,
  mas também $137 \equiv 4 \pmod{19}$ pois $137-4 = 133 = 7\times 19$.
  Também podemos usar números negativos: $-19 \equiv 2 \pmod 3$
  pois $3 \mid -19 - 2 = -21 = 3 \times (-7)$.
\end{example}

Relembramos as principais propriedades sobre a congruência
e a aritmética modular na proposição a seguir. As demonstrações
podem ser encontradas em \cite[Lema 1.1]{tnumgugu}.
\begin{proposition}\label{prop:prarit}
  Sejam $m \in \NN$, $a,b,c,d \in \ZZ$. Valem as propriedades
  a seguir:
  \begin{itemize}
    \item[i)] $a \equiv a \pmod m$,
    \item[ii)] Se $a \equiv b \pmod m$, então $b \equiv a \pmod m$,
    \item[iii)] Se $a\equiv b \pmod m$ e $b \equiv c \pmod m$, então
    $a \equiv c \pmod m$.
    \item[iv)] Se $a\equiv b \pmod m$ e
        $c \equiv d \pmod m$, então $a+c \equiv b+d\pmod m$
    \item[v)] Se $a\equiv b \pmod m$ e $c \equiv d \pmod m$,
        então $ac \equiv bd\pmod m$
  \end{itemize}
\end{proposition}

As três primeiras propriedades implicam que a congruência
módulo $m$ é uma relação de equivalência, enquanto as últimas duas
dizem que a congruência módulo $m$ respeita a aritmética.

Se $a = qm +r$ é a divisão euclidiana de $a$ por $m$,
então $m \mid a-r$ e portanto $a\equiv r \pmod m$. Assim, dois
números com o mesmo resto na divisão por $m$ são congruentes.
É um exercício simples verificar que a recíproca também
vale; isto é, se $a \equiv b\pmod m$ então $a$ e $b$ deixam
o mesmo resto na divisão por $m$. Esses resultados
implicam que podemos trabalhar com congruência
módulo $m$ nos limitando aos possíveis restos 
da divisão por $m$: $0,1,2,\cdots,m$. Esse é um
fato importante ao se trabalhar com aritmética modular
no \sage.

% Considere a divisão euclidiana de $a$ e $b$ por $m$, 
% digamos $a = q_1m+r1$ e $b = q_2m+r_2$. Note que se $a$ e $b$
% deixam o mesmo, isto é, $r_1 = r_2$, então $b-a = (q_2-q_1)m$,
% portanto $m \mid b-a$, logo $a \equiv b \pmod m$.
% A  recíproca também é válida: se $a \equiv b \pmod m$, então
% $m \mid b-a = (q_2-q_1)m + (r_2-r_1) \therefore m\mid r_2 - r_1$
% e, como $0\leq r_1,r_2 < m$ devemos ter $r_1 = r_2$. Também
% é óbvio que todo número é equivalente.

O $\Sage$ tem diversas ferramentas para se trabalhar com aritmética modular.
A forma mais primitiva é usar o operador \ils{\%} do próprio Python; a
expressão \ils{a \% m} retorna o resto da divisão de $a$ por $m$,
o resultado é, como vimos acima, congruente a $a$ módulo $m$.

\begin{sageinput}
3 % 2
\end{sageinput}
\begin{sageoutput}
1
\end{sageoutput}

O operador \ils{\%} deve ser usado com cuidado pois é um operador
que retorna um inteiro da classe \ils{Integer} (ou \ils{int} no Python),
de forma que \ils{3 \% 5 == 11 \% 8} retorna \ilso{True}, no entanto,
da forma como definimos, $3 \pmod 5 = 11 \pmod 8$ não faz sentido. Além
disso, o \ils{\%} é um operador binário, então devemos
ter cautela ao pensar em \ils{\% m} como um $\pmod m$ no final
de uma expressão: não podemos escrever, por exemplo,
$2 + 2 \pmod 3$ como \ils{2 + 2 \% 2}. Teste e veja o resultado!

Na prática existe outra forma de se trabalhar com aritmética
modular no \sage, a função \ils{mod} e o método \ils{.mod},
que funcionam de formas ligeiramente distintas.
Vejamos alguns exemplos, compare com os resultados do Exemplo
\index[sage]{\ils{.mod}} \index[sage]{\ils{mod}}
\ref{ex:modbas}:
\begin{sageinput}
mod(137,19)
mod(137,19) == mod(4,19)
(-19).mod(3)
\end{sageinput}
\begin{sageoutput}
4
True
2
\end{sageoutput}

Note que, quando $m>0$, tanto o \ils{mod} quanto o \ils{\%}
retornam o resto (se $m<0$ o comportamento do \ils{\%} 
é diferente). Existe uma diferença fundamental entre
no resultado da função \ils{mod}. Enquanto o método \ils{.mod}
e o operador \ils{\%} retornam um número inteiro, o resultado
da função \ils{mod} é um \emph{inteiro módulo $m$}.
\begin{sageinput}
parent(mod(5,13))
\end{sageinput}
\begin{sageoutput}
Ring of integers modulo 13
\end{sageoutput}

\paragraph{Inteirmos módulo $m$} 
O tipo de objeto retornardo pela função \ils{mod} é chamado
de \emph{inteiro módulo $m$}. A ideia é que, como todo
inteiro é congruente a um dos restos $0,1,\dots,m-1$ módulo $m$
podemos \emph{ignorar} os demais inteiros e trabalhar apenas
com os restos, identificando os inteiros fora desse intervalo
com seu resto na divisão por $m$. Formalmente, o que estamos
usando é o fato da congruência ser uma
\emph{relação de equivalência}, \index{Relação de equivalência}
de acordo com as três primeiras propriedades da Proposição
\ref{prop:prarit}. O conjunto dos inteiros módulo $m$ seria
portanto o conjunto quociente de $\ZZ$ por essa relação
de equivalência. Em geral denotamos tal conjunto por
$\ZZ_m$ ou $\ZZ/m\ZZ$. No \sage, podemos definir
o conjunto dos inteiros módulo $m$ usando a função \ils{Zmod},
outro nome para essa função é  \ils{IntegerModRing}.
A seguir definimos $\ZZ_7$ e listamos seus elementos.
\index[sage]{\ils{Zmod}} \index[sage]{\ils{IntegerModRing}}
\begin{sageinput}
ZZ7 = Zmod(7)
print(ZZ7)
print(list(ZZ7))
\end{sageinput}
\begin{sageoutput}
Ring of integers modulo 7
[0, 1, 2, 3, 4, 5, 6]
\end{sageoutput}
Apesar de serem exibidos como inteiros, formalmente
os elementos de $\ZZ_m$ são 
classes de equivalência, em geral
denotados por $[k]$, $\bar k$ ou $k + m\ZZ$,
e a comparação com o \ils{==} desses elementos é,
na verdade, a congruência $\cdot \equiv \cdot \pmod m$.
 Para manipular inteiros como
elementos desse conjunto podemos usar a coerção:
\begin{sageinput}
ZZ3 = Zmod(3)
a = 13
b = ZZ3(a)
print(a, parent(a))
print(b, parent(b))
\end{sageinput}
\begin{sageoutput}
13 Integer Ring
1 Ring of integers modulo 3
\end{sageoutput}

A linha $3$ tem o mesmo efeito de \ils{b = mod(a,3)}.
A aritmética entre elementos de $\ZZ_m$ funciona como 
esperado pela congruência módulo $m$, podemos
fazer as operações tanto usando a coerção para \ils{Zmod(m)}
ou usando o \ils{mod}
\begin{sageinput}
ZZ7 = Zmod(7)
(a,b) = (4,5)
print("Usando o mod:", mod(a,7) + mod(b,7))
print("Usando a coercao:", ZZ7(a) + ZZ7(b))
print("Sao iguais:", mod(a,7) + mod(b,7) == ZZ7(a) + ZZ7(b))
\end{sageinput}
\begin{sageoutput}
Usando o mod: 2
Usando a coercao: 2
Sao iguais: True
\end{sageoutput}
E, de fato, $4 + 5 \equiv 2 \pmod 7$ \footnote{E se você
fizesse a conta usando \ils{mod(a+b,7)}? O resultado seria
o mesmo! Na verdade, esse fato é essencial ao trabalharmos
com $\ZZ_m$, já que garante que a soma herdada dos inteiros
está \emph{bem definida}, o que permite trabalhar com
$\ZZ_m$ como um grupo.}.
Se, ao invés de usar o \ils{mod} ou a coerção tívessemos
usado o método \ils{.mod} ou o operador \ils{\%}, 
os objetos retornados seriam diferentes, no entanto,
ao compará-los com o \ils{==} o resultado seria verdadeiro
, por exemplo, \ils{mod(3,7) == 3 \% 7} retorna \ilso{True}
\footnote{Para os conhecedores de Python: Isso também
é feito no Python, por exemplo, \ils{float(1) == int(1)}
retorna \ilso{True}}. No entanto os métodos disponíveis
para os dois resultados são bastante distintos.

Como há apenas uma quantidade finita de elementos em cada
$\ZZ_m$, podemos listar o resultado de todas as somas possíveis
entre seus elementos. Isso pode ser feito usando o
o método \ils{.addition\_table}.
\index[sage]{\ils{.addition\_table}}
\begin{sageinput}
ZZ7 = Zmod(7)
ZZ7.addition_table(names='elements')                                                               
\end{sageinput}
\begin{sageoutput}
+  0 1 2 3 4 5 6
 +--------------
0| 0 1 2 3 4 5 6
1| 1 2 3 4 5 6 0
2| 2 3 4 5 6 0 1
3| 3 4 5 6 0 1 2
4| 4 5 6 0 1 2 3
5| 5 6 0 1 2 3 4
6| 6 0 1 2 3 4 5
\end{sageoutput}
O parâmetro opcional \ils{names='elements'} serve para
exibir o resultado de forma mais natural, tente retirá-lo
e interpretar o resultado. Ao usar
a função \ils{show} para exibir o resultado, obtemos a 
Tabela \ref{tab:Z7}

\begin{table}[h]
$$
\newcommand{\Bold}[1]{\mathbf{#1}}{\setlength{\arraycolsep}{2ex}
\begin{array}{r|*{7}{r}}
\multicolumn{1}{c|}{+}&0&1&2&3&4&5&6\\\hline
{}0&0&1&2&3&4&5&6\\
{}1&1&2&3&4&5&6&0\\
{}2&2&3&4&5&6&0&1\\
{}3&3&4&5&6&0&1&2\\
{}4&4&5&6&0&1&2&3\\
{}5&5&6&0&1&2&3&4\\
{}6&6&0&1&2&3&4&5\\
\end{array}}
$$
\caption{Tabela de adição do $\ZZ_7$}
\label{tab:Z7}
\end{table}

\begin{exercise}
\label{ex:grupoz7}
  Observe, da Tabela \ref{tab:Z7}, que: \textbf{1)} o $0$ é um
  elemento neutro para a adição, isto é,
  $0 + a \equiv a + 0 \equiv a \pmod 7$; \textbf{2)} para todo
  $a \in \ZZ_7$, existe um $b \in \ZZ_7$ tal que $a+b \equiv b+a \equiv 0
  \pmod 7$, em outras palavras, todo elemento tem um
  inverso aditivo em $\ZZ_7$; e \textbf{3)} a soma
  em $\ZZ_7$, sendo herdada de $\ZZ$, é associativa.
  Verifique que essas três propriedades valem para todo
  os conjuntos $\ZZ_m$. Dizemos que $\ZZ_m$ é um \emph{grupo}
  \index{Grupo}
  com a adição módulo $m$.
\end{exercise}



Uma das aplicações da aritmética modular é obter
informações sobre problemas envolvendo números inteiros
o analisando
em um ambiente onde há finitas possibilidades a serem
testadas.A ideia
é que, se alguma igualdade vale entre expressões
envolvendo inteiros, então ela também vale quando
reduzimos a expressão módulo $m$.  Vejamos um exemplo
em que a aritmética
modular resolve uma equação diofantina. 

\begin{example}
 Considere
  a equação $x^2 + y^2 = 4xy + 3$. 
  No entanto, suponha que exista uma solução $(x,y)$ para tal equação,
  devemos ter, portanto, $x^2 + y^2 \equiv 4xy + 3 \pmod 4$.
  Analisemos os casos, sabemos que existem $4$ restos possíveis
  para um inteiro na divisão por $4$, assim $x \equiv 0,1,2
  \text{ ou } 3 \pmod 4$, portanto $x^2 \equiv 0^2, 1^2, 2^2
  \text{ ou }  3^2 \pmod 4$, assim $x^2 \equiv
  \pmod 4$. De forma análoga $y^2 \equiv  0 \text{ ou } 1 \pmod 4$,
  assim, para o lado esquerdo, temos
  $$
    x^2 + y^2 \equiv (0 \text{ ou } 1) + (0 \text{ ou } 1) \equiv 0, 1
    \text{ ou } 2 \pmod 4
  $$
  Por outro lado, no lado direito, como $4xy$ é múltiplo de $4$,
  temos que $4xy \equiv 0 \pmod 4$, assim
  $$
    4xy + 3 \equiv 3 \pmod 4
  $$
  Como $3 \not \equiv 0, 1 \text{ ou } 2 \pmod 4$ concluímos que
  a suposta solução não pode existir. Assim, $x^2 + y^2 = 4xy + 3$
  não possui soluções em $\ZZ$. Essa equação define uma
  hipérbole em no plano real, o que provamos é que não há
  nenhum ponto nessa hipérbole com ambas as coordenadas inteiras.
\end{example}

Ainda que a análise de uma
equação módulo $m$ não produza uma inconsistência
que permita concluir a não existência de soluções, como acima,
por vezes podemos
subtrair dela alguma informação (veja o exercício \ref{ex:ecdiofcong}).

Vale destacar que, no caso geral, 
a análise da equação módulo $m$ não resolve completamente
o problema: 
existem equações que possuem soluções módulo $m$ para todo $m$
mas não possuem soluções inteiras. 
O procedimento de analisar soluções para equações nos inteiros
estudando as reduções módulo $m$ é estudado pelo Princípio
de Hasse, veja \cite{aitken2011counterexamples} para
uma descrição detalhada.

\section{Invertíveis módulo $m$}
\label{sec:invmodm}
Voltamos às propriedades da aritmética modular. Note que
não há, na Proposição \ref{prop:prarit}, uma propriedade análoga ao
\emph{cancelamento} que é permitido para
números inteiros: se $ac = bc$ com $c \neq 0$, então
$a = b$. Essa propriedade não á válida em geral na aritmética
modular, por exemplo
$$
  4\times 3  = 12 \equiv 0 \equiv 18 = 3\times 6 \pmod 6
\text{ mas } 4 \not\equiv 6 \pmod 6
$$
ou seja, não podemos
cancelar o $3$ em ambos os lados da congruência. Em alguns
casos o cancelamento pode, de fato, ser feito. A proposição
a seguir mostra o caso geral, a demonstração
fica a cargo ao leitor.

\begin{proposition}
  Sejam $a,b,c \in \ZZ$ e $m \in \NN$. Se $ac \equiv bc
  \pmod m$, então $a \equiv b \pmod{\frac{m}{d}}$, onde
  $d = \mdc(m,c)$. Em particular, se $\mdc(m,c) = 1$,
  então $a \equiv b \pmod m$.
\end{proposition}

A validade de cancelar um $c$ numa equação em
congruência está intimamente ligado com o
conceito de invertível módulo $m$.  Dizemos que
$c \in \ZZ$ é invertível módulo $m$ se existe $c' \in \ZZ$
tal que $cc' \equiv 1 \pmod m$.  O inteiro
$c'$ é dito \emph{inverso} de $c$ módulo $m$.
\index{Inverso módulo $m$}
Por exemplo,
$c = 3$ é invertível módulo $4$ pois $c' = 3$
satisfaz $cc' = 3\times 3 \equiv 9 \equiv 1 \pmod m$.
Por outro lado $2$ não é invertível módulo $4$ (verifique isso!).

Verifiquemos se $7$ é invertível módulo $11$. 
Para isso basta calcular $7d \pmod{11}$ para todos
os resíduos $d$ módulo $11$ possíveis:
\begin{sageinput}
c = 7
for d in [0..10]:
  print("7*{} = ".format(d), mod(c*d,11), "mod 11")
\end{sageinput}
\begin{sageoutput}
7*0 =  0 mod 11
7*1 =  7 mod 11
7*2 =  3 mod 11
7*3 =  10 mod 11
7*4 =  6 mod 11
7*5 =  2 mod 11
7*6 =  9 mod 11
7*7 =  5 mod 11
7*8 =  1 mod 11
7*9 =  8 mod 11
7*10 =  4 mod 11
\end{sageoutput}
Como $7\times 8 \equiv 1 \pmod{11}$, temos que $7$ é invertível
módulo $11$. Implemente o caso geral no exercício a seguir.

\begin{exercise}
\label{ex:invers}
  Crie um código que encontre, usando a definição,
  todos os invertíveis módulo um $m$ dado (Para cada $c$,
  percorra todos os resíduos possíveis $d = 0, 1, \dots, m-1$,
  verificando se $cd \equiv 1 \pmod m$). 
  \begin{itemize}
    \item[a)] Teste o seu código para alguns valores distintos de $m$. 
    \item[b)] Se um $c$ é invertível módulo $m$, quantas
    inversas ele possui?
    \item[c)] Você consegue encontrar algum padrão
    nos invertíveis módulo $m$, $m$ fixo?
    \item[d)] Você consegue encontrar algum padrão na quantidade
    de invertíveis módulo $m$? Teste primeiramente valores primos
    para $m$, e,  em seguida, potências de primos.
    \item[e)] Use o método \ils{.multiplication\_table}
    para exibir a tabela de multiplicação do anel de inteiros módulo
    $m$, como fizemos com a adição.
  \end{itemize}
\end{exercise}

A busca por invertíveis e inversas que implementamos
através da força bruta no exercício anterior pode ser
reduzida a um calculo já conhecido nosso.

\begin{proposition}
\label{prop:inversa}
  Seja $m \in \NN$ fixado. Um inteiro $c \in \ZZ$ é
  invertível módulo $m$ se e somente se $\mdc(a,m) = 1$.
  Além disso, se $x,y \in \ZZ$ são tais que
  $cx+my = 1$, então $x$ é uma inversa para $c$
  módulo $m$.
\end{proposition}
\begin{proof}
  Pelo Teorema \ref{thm:bezout}, se $c$ e $m$ são coprimos,
  existem $x,y \in \ZZ$ tais que $cx + my = 1$, reduzindo
  essa igualdade módulo $m$ temos que $cx \equiv 1 \pmod m$,
  portanto $c$ é invertível módulo $m$ e $x$ é sua 
  inversa. Por outro lado, se $c$ é invertível módulo $m$ 
  existe $c' \in \ZZ$ tais que $cc' \equiv 1 \pmod m$, portanto
  $m \mid cc' - 1$, logo existe $k \in \ZZ$ tal que
  $km = cc' - 1 \therefore cc' - km = 1$. Suponha agora
  que $\mdc(m,c)\neq 1$,
  então existe um primo $p$ tal que $p\mid c$ e $p\mid m$,
  portanto $p \mid cc'-km = 1$, absurdo.
\end{proof}

Segue então que a função \ils{xgcd}, que implementa o
algorito de Euclides estendido, que vimos na Seção
\ref{sec:mdc}, é suficiente para o estudo dos invertíveis módulo
$m$, já que retorna o $\mdc$ e o par $x$ e $y$.
\begin{sageinput}
(c,m) = (7,11)
print("d =", d)
print("d (mod m) =", mod(d,m))
print("{}*{} = ".format(mod(d,m),c), mod(d*c,m))
\end{sageinput}
\begin{sageoutput}
d = -3
d (mod m) = 8
8*7 =  1
\end{sageoutput} 
Como vimos acima, eventualmente o $x$ retornado pelo
algoritmo de Euclides estendido está fora
do intervalo $\{0,1,2,\cdots, m-1\}$, no 
entanto basta tomar o resíduo congruente nesse intervalo.


\begin{exercise}
  Resolva novamente o Exercício \ref{ex:invers}
  usando o \ils{xgcd}. 
\end{exercise}

Podemos encontrar a inversa usando o método \ils{.inverse\_mod}
disponível para os inteiros, o resultado também será
um inteiro.
\begin{sageinput}
7.inverse_mod(11)
print(parent(7.inverse_mod(11)))
\end{sageinput}
\begin{sageoutput}
8
Integer Ring
\end{sageoutput}
Para se manter no mundo dos inteiros módulo $m$,
podemos calcular a inversa com o \ils{\^{}(-1)}
\begin{sageinput}
Z11 = Zmod(11)
a = Z11(7)
print(a^(-1))
print(parent(a^(-1)))
\end{sageinput}
\begin{sageoutput}
8
Ring of integers modulo 11
\end{sageoutput}

O subconjunto de $\ZZ_m$ formado pelos invertíveis módulo $m$ é
denotado por $\ZZ_m^\ast$ e chamado de \emph{grupo multiplicativo dos
inteiros módulo $m$}
\index{Grupo!multiplicativo dos inteiros módulo $m$}
\footnote{Para os conhecedores de álgebra: Se $A$ é um anel
comutativo com identidade
então $A^\ast$ denota o grupo das unidades, isto é, dos invertíveis,
de $A$. O conjunto dos inteiros módulo $m$ é um dos primeiros
exemplos de aneis estudados.}. Já vimos o conceito de grupo
no Exercício \ref{ex:grupoz7}. Não é difícil verificar que
$\ZZ_m^\ast$ também satisfaz as três propriedades lá mencionadas.
Explicitamos no exercício a seguir.

\begin{exercise}
  Seja $m  \in \NN$ fixado e $\ZZ_m^\ast$ o conjunto dos
  invertíveis módulo $n$.  Verifique que 
  \begin{itemize}
    \item[a)] $1 \in \ZZ_m^\ast$ e $1\times n \equiv n \times 1 \equiv 1
        \pmod m$.
    \item[b)] Para todo $n \in \ZZ_m^\ast$ existe $n'$ tal que
    $nn' \equiv n'n \equiv 1 \pmod m$.
    \item[c)] A  multiplicação módulo $m$ é associativa, isto é,
    para quaisquer $n_1,n_2,n_3 \in \ZZ_m^*$, vale $n_1(n_2n_3)
    \equiv (n_1n_2)n_3 \pmod m$. 
  \end{itemize}
\end{exercise}

Justifica-se assim o uso do termo grupo multiplicativo para
$\ZZ_m^*$. O \sage sabe trabalhar com grupos e podemos, inclusive,
definir o grupo dos invertíveis módulo $m$ usando o
método \ils{unit\_group}. No entanto, esse grupo será definido
no sage de forma abstrata e não será útil aos nossos propositos,
de forma que continuaremos trabalhando com inteiros e inteiros
módulo $m$.

Note que $\ZZ_m^*$, sendo subconjunto de $\ZZ_m$,
também é um conjunto de classes de equivalência, de forma
que, apesar de existirem infinitos inteiros que são
invertíveis módulo $m$ estamos considerando apenas os
que não são congruentes entre si. Podemos então
listar os elementos em $\ZZ_m^*$, usando
que $\ZZ_m = \{0,1,2,\dots,m-1\}$ (tecnicamente os 
elementos são classes, não inteiros). Pela Proposição
\ref{prop:inversa} basta nos limitarmos aos 
coprimos com $m$. Para isso usamos uma compreensão de listas
com um condicional.
\begin{sageinput}
m = 16
[n for n in [0..m-1] if gcd(n,m) == 1]
\end{sageinput}
\begin{sageoutput}
[1, 3, 5, 7, 9, 11, 13, 15]
\end{sageoutput}
Matematicamente, o código na compreensão de listas
é precisamente $$\{n \in \{0,1,\dots,m-1\} \mid \mdc(n,m) = 1\},$$
o mesmo conjunto que usamos para a definição da função $\varphi$
na Seção \ref{sec:funarit}! Concluímos portanto que existem
$\varphi(m)$ invertíveis módulo $m$, a menos de congruencia,
ou seja, $|\ZZ_m^\ast| = \varphi(m)$. Mencionamos que o $\varphi$
pode ser calculada usando a função \ils{euler\_phi}.


\begin{tcolorbox}[colback=red!5,colframe=red!75!black,title=Objetivo]
 Tentar escrever os resultados envolvendo $\ZZ_m^\ast$
em linguagem não algébrica
\end{tcolorbox}
\section{Fermat, Euler e Wilson}



%%%%%%%%%%%%%%%%%%%%%%%%%%%%%%%%%%%%%%%%%%%%%%%%%%%%%%%%%%%%%%%%
%%%%%%%%%%%%%%%%%%%%%%%%%%%%%%%%%%%%%%%%%%%%%%%%%%%%%%%%%%%%%%%%
%%%%%%%%%%%%%%%%%%%%%%%%%%%%%%%%%%%%%%%%%%%%%%%%%%%%%%%%%%%%%%%%

\section{Teorema Chines dos Restos}


\section{Criptografia (como subseção do Explore?)}
\label{sec:cripto}

\begin{itemize}
  \item exponenciação binária
  \item unidades, Teorema de Euler
  \item Teorema chinês dos restos
  \item raizes primitivas (fórmula de \cite{disquisitiones})
\end{itemize}


\section{Explore!}


\subsection{Algoritmo AKS - Primos em \textbf{P}}

Primos em tempo polinomial \cite{agrawal2004primes}


\subsection{Teste de Miller-Rabin}

\subsection{Encontrando $a,b \in \ZZ$ tais que $p = a^2 + b^2$}

\section{Exercises}

\begin{exercise}\label{ex:ecdiofcong}
  Considere a equação diofantina $x^4 + y^4 = 3z^4$.
  \begin{itemize}
    \item[a)] Liste todos os resíduos possíveis de
    $n^4 \pmod 5$ (Dica: use compreensão de listas
    e a função \ils{mod})
    \item[b)] Prove que se $(x,y,z)$ é solução da equação,
    então $x \equiv y \equiv z \equiv 0 \pmod 5$.
    \item[c)] Tome $5^n$ a maior potência de $5$ dividindo
    simultaneamente $x$, $y$ e $z$, fatore $5^n$ na
    equação inicial e encontre uma equação
    $x'^4 + y'^4 = 3z'^4$ onde algum dos $x',y',z'$ não
    é múltiplo de $5$.
    \item[d)] Conclua o argumento mostrando que a equação
    inicial não tem solução.
  \end{itemize}
\end{exercise}

\begin{exercise}
  Vimos na Seção \ref{sec:invmodm} como manipular inversas
  módulo $m$. Lá comentamos que, se definirmos o grupo
  $\ZZ_m$, o \sage consegue trabalhar com o grupo dos
  invertíveis módulo $m$, o $\ZZ_m^*$.
  \begin{itemize}
    \item[a)] Tome $m = 8$ e $p = 11$. Crie \ils{A} e \ils{B}
    como $\ZZ_m$ e $\ZZ_p$, respectivamente.
    \item[b)] Use o método \ils{.\_group} para obter o grupo
    dos invertíveis módulo $m$ e módulo $p$. Explore os
    grupos criados com o método \ils{.multiplication\_table}.
    \item[c)] Interprete o resultado do método \ils{.unit\_gens}
    nos aneis \ils{A} e \ils{B}. (A discussão sobre raízes
    primitivas pode ajudar.)
  \end{itemize}
\end{exercise}