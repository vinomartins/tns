\chapter{Equações Diofantinas}
\label{chap:eqdiof}

\begin{itemize}
  \item o que são equações diofantinas
  \item exemplos famosos 
\end{itemize}

\section{Equação Linear no caso geral $aX+bY = c$}
\label{sec:dioflinear}
Vimos, no Teorema \ref{thm:bezout}, que dados
$a,b \in \ZZ$ não ambos nulos existem
$x,y \in \ZZ$ tais que $ax+by = c$ quando
$c = \mdc(a,b)$. Nessa seção
discutiremos a equação diofantina linear
no caso mais geral $aX + bY = c$ com $a,b,c \in \ZZ$. 

Nos preocupamos primeiramente com a existência
de soluções. Note que, como $\mdc(a,b)$ divide
$a$ e $b$, então $\mdc(a,b)\mid ax+by$ para
quaisquer $x,y \in \ZZ$, portanto, se $ax+by = c$,
então $\mdc(a,b)\mid c$. Segue que a equação
$aX + bY = c$ não possui soluções se
$\mdc(a,b)\nmid c$. Por outro lado, se $c$ 
é divisível por $\mdc(a,b)$, digamos
$c = \mdc(a,b)\tilde{c}$ para algum $\tilde{c} \in \ZZ$. 
Pelo teorema de Bézout \ref{thm:bezout}, existem
$x,y \in \ZZ$ tais que $ax+by = \mdc(a,b)$,
portanto $\tilde{x} = x\tilde{c}$ e
$\tilde{y} = y\tilde{c}$ satisfazem
$$
a\tilde{x} + b\tilde{y} = ax\tilde{c} + by\tilde{c}
 = (ax+by)\tilde{c} = \mdc(a,b)\tilde{c} = c.
$$
Portanto, $aX + bY = c$ tem solução se e somente se
$\mdc(a,b) \mid c$. Além disso é possível provar
que a solução não é única. Na verdade, se existe uma
solução, existem infinitas delas. O Teorema a seguir
resume os fatos importantes sobre a equação
diofantina linear.
\begin{theorem}
\label{thm:eqdlineargeral}
  Sejam $a,b,c \in \ZZ$ com $a$ e $b$ não ambos nulos. Denote
  $d = \mdc(a,b)$. A equação diofantina linear $aX + bY = c$ tem
  solução se e somente se $c\mid d$. Se $(x_0,y_0) \in \ZZ^2$ é
  uma solução, então todas as soluções são da forma
  $(x_0 + tb/d,y_0 - ta/d)$ com $t \in \ZZ$. Além disso, se
  supormos que $a,b \in \NN$ são coprimos e $c>ab-a-b$, então
  a equação admite soluções com $x$ e $y$ positivos.
\end{theorem}

\begin{sageinput}
input
\end{sageinput}
\begin{sageoutput}
output
\end{sageoutput}


\begin{tcolorbox}[colback=red!5,colframe=red!75!black,title=My nice heading]
This is another \textbf{tcolorbox}.
\tcblower
Here, you see the lower part of the box.
\end{tcolorbox}

\section{Ternos Pitagóricos}
\begin{itemize}
  \item gerando soluções com $a,b$ ou $c$ dados.
  \item contando soluções?
  \item versão com matrizes
\end{itemize}

\section{Equação de Pell}
\begin{itemize}
  \item gerando soluções a partir de uma inicial
  \item versão com matrizes
\end{itemize}

\section{Outras}
\begin{itemize}
  \item ver livro sobre diofantinas
  \item exemplo do numberphile
\end{itemize}


\section{Explore!}

\section{Exercises}

\begin{exercise}
  Usando a afirmação final do Teorema \ref{thm:eqdlineargeral}
  e o exercício \ref{ex:eqdlinearsolpos}, crie uma função que,
  para $a$ e $b$ naturais dados, retorne os valores de $c$ para os quais
  existem soluções positivas para $aX + bY = c$. (O resultado
  deve ser uma lista de valores e uma constante $k$ para os quais $c\geq k$
  também satisfaz a condição desejada.)
\end{exercise}